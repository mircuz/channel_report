\section{Scaling Performance of $256^{3}$ problem}
The medium sized problem shows better scaling performances compared to the small ones.
This behavior, in which bigger problems provide better scaling capabilities, has been highlighted and discussed by many authors in past.
After a little preamble about scaling let us dig into results.\\
\par
A slab decomposed algorithm provide gains of $\mathcal{O}(10)$ in terms of execution times, less than a pencil decomposed algorithm, but with better results for small processors grid. In fact, as depicted in figure~\ref{1281} the 1D decomposition curve achieve lower execution times than the 2D ones, until 8 cores.\\
Passed 8 cores the pencil decomposition prevails, reaching speedup factors above 100, with time savings in the order of magnitude of $\mathcal{O}(100)$ with respect to the single core runtime.
In the figure~\ref{1283} is possible to see the efficiencies achieved by the two methods, running on 64 threads per processor. It is important to denote the behavior of the pencil decomposed algorithm, which, until 8 cores in use, exhibits high scaling efficiency.
Passed 8 cores, to recover high efficiency we must decrease the number of threads per processor.