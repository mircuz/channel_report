\chapter{Code Structure}
\section{Spatial discretization along homogeneous directions}
Our solver is based on a Fourier approach. Among the advantages of such approach we face the possibility to expansion of the unknown functions in terms of truncated Fourier series in the homogeneous directions. For example the wall-normal component $v$ of the velocity vector is represented as:
\begin{equation}
v(x,y,z,t) = \sum_{h=-nx/2}^{+nx/2} \sum_{l=-nz/2}^{+nz/2} \hat{v}_{hl}(y,t) e^{i\alpha x}e^{i \beta z}
\end{equation}
where:
\begin{equation}
\alpha = \frac{2\pi h}{L_{x}} = \alpha_{0} h; \quad \beta= \frac{2 \pi l}{L_{z}} = \beta_{0}l
\end{equation}

$h$ and $l$ are integer indexes corresponding to the streamwise and spanwise direction respectively, and $\alpha_{0}$ and $\beta_{0}$ are the fundamental wavenumbers in these directions, defined in terms of the streamwise and spanwise lengths $L_{x} = {2\pi}/{\alpha_{0}} $ and $L_{z} = {2 \pi}/{\beta_{0}}$ of the computational domain. The computational parameters given by the streamwise and spanwise lenght of the computational domain, $L_{x}$ and $L_{z}$ , and the truncation of the series, $nx$ and $nz$, must be chosen so as to miminize computational errors. For further details regarding the proper choice of a value of $L_{x}$ see~\cite{QuadrioMaurizio2003Issi}.

The convolutions required to solve the equations~\ref{curl:momentum:y} and~\ref{normal:velocity} are computationally expensive if carried out in the frequency domain. The same evaluation can be performed efficiently by first transforming the three Fourier components of velocity back in physical space, multiplying them in all six possible pair combinations and eventually retransforming the results into the Fourier space. Fast Fourier Transform algorithms are used to move from Fourier to physical space and viceversa. The aliasing error is removed by expanding the number of modes by a factor of at least $3/2$ before the inverse Fourier transforms, to avoid the introduction of spurious energy from the high-frequency into the low-frequency modes during the calculation.



\section{Finite difference scheme}


\subsection{Compute of the finite difference coefficients}




\section{Time Discretization}





\section{Domain Decompositions}
The engine of Quadrio and Luchini described into~\cite{cpl:presentazione} works per \emph{y-slabs}, as shown in figure~\ref{domain_decomp},
\begin{figure}
\centering
\includegraphics[width=0.8\textwidth]{grafici/decomp_dominio_cpl}
\caption{Original domain decomposition in case of 4 processors}
\label{domain_decomp}
\end{figure}
 allowing to perform convolutions and Fourier transformations locally on each processor, avoiding the cost of non-local transposition for the velocity array and the non-linear terms ones. Such implementation, denominated pipelined-linear-system (PLS), lead to a minimum of communication, in fact this approach require to send and receive only the values stored in the two upper and lower boundary cells of the local decomposition, in order to provide the data required by the fourth-order finite difference scheme.
 \par
 Using the PLS approach the number of bytes exchanged in a three step Range-Kutta method are:
 \begin{equation}
 D_{t} = 3 \times 8 \times (p-1) \frac{3}{2} \times \frac{nx}{p} \frac{nz}{p} \times ny \times 18
 \label{exchange:data:cpl}
 \end{equation}
 where:
 \begin{description}
  \item[3] takes into account the number of time steps;
  \item[8] for counting the bytes;
  \item[$\mathbf{(p-1)}$] is the number of nodes across which the exchange take place;
  \item[ $\mathbf{\frac{3}{2}}$ ] corresponds to the expansion in horizontal modes required by the dealiasing process;
  \item[ $\mathbf{\frac{nx}{p} \times \frac{nz}{p}}$] is the grid portion, for each plane, to exchange with the others nodes;
  \item[18] due to the 3 velocities plus the 6 products to be exchanged twice, before and after the FFT;
  \item[ny] takes into account the number of planes to be exchanged.
\end{description}
Further details about the PLS communications are available in \cite[\nopp chapter 4.2]{ns:quadrio}. \\
 \par
Although efficient for small processors grid, the performances of this approach falls quickly whether the processors number becomes comparable with \emph{ny}.  Furthermore the code structure limit the number of parallel process to be just a fraction of the \emph{ny} extension.
\par
To avoid such limitations and increase the number of parallel processes we decided to move from PLS approach to something different.
We have identified two possible solutions:
\begin{description}
  \item employ \textbf{slab} decomposition along x or z axis;
  \item employ \textbf{pencil} decomposition.
\end{description}
Both implementation require extensive use of the MPI paradigm and, possibly, a library to handle such decompositions.
We opted to employ OpenMPI\cite{openmpi} for what concern the MPI paradigm, in particular we entrust to the well established OpenMPI version 3.1, release 3.1.3.
The ideas behind the choice of such library rely on the fact the OpenMPI is released behind BSD license\cite{bsd:license}, it is designed to group different MPI implementation, avoiding fragmentation and forking problem\cite{faq:openmpi} and, although less optimized on proprietary fabric such as Intel Omni-Path fabric\cite{intel:omnipath}\cite{intel:intelmpivsopenmpi}, is wider spread.


 while, for the decomposition, we entrusted in a new library, released by the 


\subsection{Slabs decomposition}


\subsection{Pencil decomposition}

