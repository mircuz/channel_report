The following chapter will show the performances obtained after the MPI integration in our code.
We have benchmarked three different problem sizes:
\begin{itemize}
\item small dimension problem $128\times 128 \times 128$;
\item medium dimension problem $512\times 512\times 512$;
\item large dimension problem $4096\times 512\times 512$.  
\end{itemize}
All problems have been tested using 1D and 2D decomposition, so that it is possible to have a comparison among this two methods.
\\
We want to highlight that all these problems exploit the Hermitian symmetry, along the streetwise axis.
\section{Testing environment}
Our test were conducted at CINECA\cite{Cineca}, an Italian academic research center, which host the $19^{th}$ most powerful supercomputers of the TOP500~\cite{top500} of November 2018 list.

We worked on Marconi\cite{marconi:specs} supercomputer, in particular on Marconi-A2 partition.
Marconi structure fuse different partition to reach the peak performance of 20 PFlop/s; in particular our partition is characterized by 3600 nodes, connected through Intel OmniPath\cite{intel:intelmpivsopenmpi} high performance network. Each node host a 68-cores Intel Xeon Phi 7250, code name Knight Landings, and about 100GB of ram. \\
The machine runs on CentOS 7.2~\cite{centos}, a Linux distribution, and our benchmark code has been compiled using GNU GCC 7.3~\cite{gcc} with OpenMPI 3.0.0~\cite{openmpi}\cite{MPI:standard3} . \\
\par
We have tested different flags during the compilation phase, trying to enable different levels of optimization and code vectorization, the most important feature of the Xeon Phi, however the best results have been achieved using:
\begin{lstlisting}
-O2 -fpic -march=native -std=c99
\end{lstlisting}
This behavior was expected since our code does not include the OpenMP\cite{openmp} features at present time, so the MIC\cite{mic} (Many Integrated Cores) architecture can not exploit such fundamental feature.



\section{Scaling performance of $128^3$ problem}

