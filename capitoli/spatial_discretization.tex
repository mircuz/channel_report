\section{Spatial discretization along homogeneous directions}
Our solver is based on a Fourier approach. Among the advantages of such approach we face the possibility to expansion of the unknown functions in terms of truncated Fourier series in the homogeneous directions. For example the wall-normal component $v$ of the velocity vector is represented as:
\begin{equation}
v(x,y,z,t) = \sum_{h=-nx/2}^{+nx/2} \sum_{l=-nz/2}^{+nz/2} \hat{v}_{hl}(y,t) e^{i\alpha x}e^{i \beta z}
\end{equation}
where:
\begin{equation}
\alpha = \frac{2\pi h}{L_{x}} = \alpha_{0} h; \quad \beta= \frac{2 \pi l}{L_{z}} = \beta_{0}l
\end{equation}

$h$ and $l$ are integer indexes corresponding to the streamwise and spanwise direction respectively, and $\alpha_{0}$ and $\beta_{0}$ are the fundamental wavenumbers in these directions, defined in terms of the streamwise and spanwise lengths $L_{x} = {2\pi}/{\alpha_{0}} $ and $L_{z} = {2 \pi}/{\beta_{0}}$ of the computational domain. The computational parameters given by the streamwise and spanwise lenght of the computational domain, $L_{x}$ and $L_{z}$ , and the truncation of the series, $nx$ and $nz$, must be chosen so as to miminize computational errors. For further details regarding the proper choice of a value of $L_{x}$ see~\cite{QuadrioMaurizio2003Issi}.

The convolutions required to solve the equations~\ref{curl:momentum:y} and~\ref{normal:velocity} are computationally expensive if carried out in the frequency domain. The same evaluation can be performed efficiently by first transforming the three Fourier components of velocity back in physical space, multiplying them in all six possible pair combinations and eventually retransforming the results into the Fourier space. Fast Fourier Transform algorithms are used to move from Fourier to physical space and viceversa. The aliasing error is removed by expanding the number of modes by a factor of at least $3/2$ before the inverse Fourier transforms, to avoid the introduction of spurious energy from the high-frequency into the low-frequency modes during the calculation~\cite{ns:quadrio}.





\section{Finite difference scheme}
The discretization of the wall-normal derivatives $D_{1}$, $D_{2}$ and $D_{4}$, required for the numerical solution of the present problem, is performed through finite difference (FD) compact schemes~\cite{finite:difference:scheme} with fourth-order accuracy over a computational molecule composed by five arbitrarily spaced grid points. We indicate here with $d_{1}^{j} (i), i = -2, \dots , 2$ the five coefficients discretizing the exact operator $D_{1}$ over five adjacent grid points centered at $y_{j}$ :
\begin{equation}
D_{1} \big( f(y) \big) \vert_{y=y_{j}} = \sum_{i=-2}^{2} d_{1}^{j} (i) f(y_{j+i})
\end{equation}
The basic idea of compact schemes can be most easily understood by thinking of a standard FD formula in Fourier space as a polynomial interpolation of a trascendent function, with the degree of the polynomial corresponding to the formal order of accuracy of the FD formula. Compact schemes improve the interpolation by replacing the polynomial with a ratio of two polynomials, i.e. with a rational function. This obviously increases the number of available coefficients, and moreover gives control over the behavior at infinity (in frequency space) of the interpolant, whereas a polynomial necessarily diverges. This allows a compact FD formula to approximate a differential operator in a wider frequency range, thus achieving resolution properties similar to those of spectral schemes~\cite{finite:difference:scheme}.
Compact schemes are also known as implicit finite-differences schemes, because they typically require the inversion of a linear system for the actual calculation of a derivative~\cite{compact:difference}\cite{finite:difference:scheme}.
 Here we are able to use compact, fourth-order accurate schemes at the cost of explicit schemes, owing to the absence of the third-derivative operator from the equations of motion. Thanks to this property, it is possible to find rational function approximations for the required three FD operators, where the denominator of the function is always the same, as highlighted first in the original Gauss-Jackson-Noumerov compact formulation exploited in his seminal work by Thomas~\cite{Thomas:coeff}, concerning the numerical solution of the Orr-Sommerfeld equations.\par
To illustrate Thomas’ method, let us consider an 4th-order one-dimensional ordinary differential equation, linear for simplicity, in the form:
\begin{equation}
D_{4}(a_{4}f) + D_{2}(a_{2}f)+D_{1}(a_{1}f)+a_{0}f = g,
\label{Thomas:coeff}
\end{equation}
Where the coefficients $a_{i}= a_{i}(y)$ are arbitrary functions of the independent variable $y$, and $g = g(y)$ is a known RHS. Let us moreover suppose that a differential operator, for example $D_{4}$, is approximated in frequency space as the ratio of two polynomials, say $\mathcal{D}_{4}$ and $\mathcal{D}_{0}$. Polynomials like $\mathcal{D}_{4}$ and $\mathcal{D}_{0}$ have their counterpart in physical space, and ${d}_{4}$ and $d_{0}$ are the corresponding FD operators. The key point is to impose that all the differential operators appearing in the example equation~\ref{Thomas:coeff} admit a representation such as the preceding one, in which the polynomial $\mathcal{D}_{0}$ at the denominator remains the same.\par
Equation~\ref{Thomas:coeff} can thus be recast in the new, discretized form:
\begin{equation}
d_{4} (a_{4}f) + d_{2} (a_{3}f) + d_{1} (a_{1}f) + d_{0} (a_{0}f) = d_{0} g,
\end{equation}
and this allows us to use explicit FD formulas, provided the operator $d_{0}$ is applied to the right-hand-side of our equations. The overhead related to the use of implicit finite difference schemes disappears, while the advantage of using high-accuracy compact schemes is retained~\cite{ns:quadrio}.





\subsection{Compute of the finite difference coefficients}
The actual computation of the coefficients $d_{0}, d_{1}, d_{2}$ and $d_{4}$ to obtain a formal accuracy of order 4 descends from the requirement that the error of the discrete operator $d_{4} / d_{0}$
decreases with the step size according to a power law with the desired exponent -4. In practice, following a standard procedure in the theory of Padé approximants~\cite{Pade:approximants}, this can be enforced by choosing a set $t_{m}$ of polynomials of $y$ of increasing degree:
\begin{equation}
\label{pade:polynomials}
t_{m} (y) = 1, y, y^{2}, \dots , y^{m},
\end{equation}
 by analytically calculating their derivatives $D_{4}(t_{m})$, and by imposing that the discrete
equation:
\begin{equation}
\label{d4:derivative}
d_{4} (t_{m}) - d_{0} (D_{4}(t_{m})) = 0
\end{equation}
is verified for the nine polynomials from $m = 0$ up to $m = 8$.
Our computational stencil contains 5 grid points, so that the unknown coefficients $d_{0}$ and
$d_{4}$ are 10. There is however a normalization condition, and we can write the equations in a form where for example:
\begin{equation}
\label{normal:condition}
\sum_{i=-2}^{2} d_{0}(i) = 1.
\end{equation}
The other 9 conditions are given by equation~\ref{d4:derivative} evaluated for $m = 0, 1, \dots 8$. We thus can set up, for each distance from the wall, a $10\times10$ linear system which can be easily solved for the unknown coefficients. 
The coefficients of the derivatives of lesser degree are derived from analogous relations, leading to two $5\times5$ linear systems once the $d_{0}$’s are known. An additional further simplification is possible. Since the polynomials~\ref{pade:polynomials} have vanishing $D_{4}$ for $m \textless 4$, thanks to the normalization condition~\ref{normal:condition} the $10\times10$ system can be split into two $5\times5$ subsystems, separately yielding the coefficients $d_{0}$ and $d_{4}$.\par
Due to the turbulence anisotropy, the use of a mesh with variable size in the wall-normal direction is advantageous. The procedure outlined above must then be performed numerically at each $y_{j}$ station, but only at the very beginning of the computations. The computer-based solution of the systems requires a negligible computing time.\par
We end up with FD operators which are altogether fourth-order accurate; the sole operator $D_{4}$ is discretized at sixth-order accuracy. As suggested in~\cite{kim_moin_moser} and~\cite{compact:difference}, the use of all the degrees of freedom for achieving the highest formal accuracy might not always be the optimal choice. In~\cite{ns:quadrio} Quadrio and Luchini attempted to discretize $D_{4}$ at fourth-order accuracy only, spending the remaining degree of freedom to improve the spectral characteristics of all the FD operators at the same time. Their search has shown however that no significant advantage can be achieved: the maximum of the errors can be reduced only very slightly, and, more important, this reduction does not carry over to the entire frequency range.\par
The boundaries obviously require non-standard schemes to be designed to properly compute derivatives at the wall. For the boundary points we use non-centered schemes, whose coefficients are computed following the same approach as the interior points, thus preserving by construction the formal accuracy of the method. Nevertheless the numerical error contributed by the boundary presumably carries a higher weight than interior points, albeit mitigated by the non-uniform discretization. 
