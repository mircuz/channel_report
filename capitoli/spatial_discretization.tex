\section{Spatial discretization along homogeneous directions}
Our solver is based on a Fourier approach. Among the advantages of such approach we face the possibility to expansion of the unknown functions in terms of truncated Fourier series in the homogeneous directions. For example the wall-normal component $v$ of the velocity vector is represented as:
\begin{equation}
v(x,y,z,t) = \sum_{h=-nx/2}^{+nx/2} \sum_{l=-nz/2}^{+nz/2} \hat{v}_{hl}(y,t) e^{i\alpha x}e^{i \beta z}
\end{equation}
where:
\begin{equation}
\alpha = \frac{2\pi h}{L_{x}} = \alpha_{0} h; \quad \beta= \frac{2 \pi l}{L_{z}} = \beta_{0}l
\end{equation}

$h$ and $l$ are integer indexes corresponding to the streamwise and spanwise direction respectively, and $\alpha_{0}$ and $\beta_{0}$ are the fundamental wavenumbers in these directions, defined in terms of the streamwise and spanwise lengths $L_{x} = {2\pi}/{\alpha_{0}} $ and $L_{z} = {2 \pi}/{\beta_{0}}$ of the computational domain. The computational parameters given by the streamwise and spanwise lenght of the computational domain, $L_{x}$ and $L_{z}$ , and the truncation of the series, $nx$ and $nz$, must be chosen so as to miminize computational errors. For further details regarding the proper choice of a value of $L_{x}$ see~\cite{QuadrioMaurizio2003Issi}.

The convolutions required to solve the equations~\ref{curl:momentum:y} and~\ref{normal:velocity} are computationally expensive if carried out in the frequency domain. The same evaluation can be performed efficiently by first transforming the three Fourier components of velocity back in physical space, multiplying them in all six possible pair combinations and eventually retransforming the results into the Fourier space. Fast Fourier Transform algorithms are used to move from Fourier to physical space and viceversa. The aliasing error is removed by expanding the number of modes by a factor of at least $3/2$ before the inverse Fourier transforms, to avoid the introduction of spurious energy from the high-frequency into the low-frequency modes during the calculation.



\section{Finite difference scheme}


\subsection{Compute of the finite difference coefficients}