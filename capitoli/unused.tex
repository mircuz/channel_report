%UNUSED

\subsection{Reynolds decomposition and time averaging}
The needing to employ a statistical approach require to express the quantities as a mean value plus fluctuations.\par
The expected value is calculated as a time averaged value and denoted with a bar, while the fluctuations with an appendix. Therefore we can express, for example the velocity, as
\begin{equation}
u(x,y,z,t) = \bar{u}(x,y,z) + u'(x,y,z,t).
\label{Reynolds:decomp}
\end{equation}
Two of the advantages of the Reynolds decomposition shown in equation~\ref{Reynolds:decomp} rely in the fact that the time average of the fluctuations is identically zero:
\begin{equation}
\frac{1}{T} \int_{0}^{T} u' dt =0,
\label{prop1}
\end{equation}
and the mean value is time-independent:
\begin{equation}
\frac{\partial \bar{u} }{\partial t} = 0.
\label{prop2}
\end{equation}
\par
Since we have to deal with non-deterministic variables we have to define their ensemble average. In accordance with theory, an ensemble average owns a Gaussian probability distribution, thanks to the central limit theorem, so it can be no longer considered as a random variable. To be consistent with the theory we should perform $n$ times the same simulation under constant conditions.
However, in fluid dynamics, it is a common practice to exploit the ergodicity property of the processes, since, usually, the experiments are conducted under the assumption of stationary flow. 
Such property affirm that, since the process is stationary, the time averaged mean is equivalent to the ensemble average:
\begin{equation}
\langle p(\mathbf{x},t) \rangle = \frac{1}{T} \int_{0}^{T} p(\mathbf{x},t) dt.
\label{ergodic}
\end{equation}
When dealing with non-stationary processes, the equation~\ref{ergodic} is not fulfilled. Yet, this definition can still be considered valid if the sampling time $T$ is chosen small compared to the time needed for the averaged properties to change significantly.
\par
Expressing the Navier-Stokes equation using~\ref{Reynolds:decomp} and imposing a time averaging of the resulting equation, carrying out the simplifications due to~\ref{prop1} and~\ref{prop2}, the Reynolds averaged Navier-Stokes equation, or RANS, arise. Such equation is characterized by the appearance of a strongly non linear term, the viscous stresses term, which describe the turbulence.
\par
Although we have an equation capable of catching turbulence development using only averaged values we are still far from solving the problem. Unfortunately, the RANS equation, suffer the closure problem. We have to rely on turbulence models to evaluate the viscous stresses term. Thus we introduce an error, which is proportional to $\sqrt{u'u'}$, which is incompatible with our requirement to depict the turbulence with the highest level of fidelity possible.





\subsection{Elements of statistics}
Before introducing the concepts of correlations and spectral analysis, useful in our studies, we must define the probability density function and cumulative distribution function. \par
The cumulative distribution function is considered as the likelihood of an event to take place. It is a monotonically increasing curves and independently from its distribution and face the following properties:
\begin{equation*}
P(-\infty) = 0, \qquad P(+\infty) = 1 \qquad and \qquad \frac{\partial P(x)}{\partial x} = p(x)
\end{equation*}
where the latter term, $p(x)$, is the so called probability density function (PDF).
\par
The PDF is fundamental in the definition of the expected value:
\begin{equation}
\langle u \rangle = \int_{-\infty}^{+\infty} u(x) p(x) dx,
\end{equation}
and in the definition of the statistical moments:
\begin{equation}
\sigma^{n} = \int_{-\infty}^{+\infty} (u(x) - \langle u \rangle)^{n} p(x)dx.
\label{statistic:momentum}
\end{equation}
Depending on the value of $n$, the equation~\ref{statistic:momentum} can define the variance ($n=2$), the skewness factor ($n=3$), the flatness factor ($n=4$) or higher statistical moments.
\par
When multiple variables are used we can recourse to the usage of the joined probability density function, $p_{x,y}(x,y)$. It exploit similar properties of the PDF and can be used to catch event characterized by the concomitancy. 
\par
It is particularly useful to determine the covariance of an event, defined as:
\begin{equation*}
\langle u_{1}u_{2} \rangle = \int_{-\infty}^{+\infty} \int_{-\infty}^{+\infty} \big( u_{1}(x_{1}) - \langle u_{1} \rangle \big) \big( u_{2}(x_{2}) - \langle u_{2} \rangle \big) p_{1,2}(x_{1},x_{2}) dx_{1} dx_{2}.
\end{equation*}
\par
Starting from the result of the last equation we can define the correlation factor, $\rho_{1,2}$, as:
\begin{equation*}
\rho_{1,2} = \frac{\langle u_{1} u_{2} \rangle}{\sqrt{\langle u_{1}^{2} \rangle \langle u_{2}^{2} \rangle}}.
\end{equation*}



\subsection{Correlations}

Keeping all the previous chapter definitions in mind and assuming ergodic processes, we can define the \emph{auto-covariance} factor as 
\begin{equation}
R_{uu}(\mathbf{x},\tau) = \langle u'(\mathbf{x},t) u'(\mathbf{x},t+\tau) \rangle.
\label{autocovariance}
\end{equation}
The equation~\ref{autocovariance} represent the correlation between the signal from a given point in space at two different instants in time. \emph{Auto-covariance} gives an idea of the time needed by the signal itself to ``forget'' its past history, in a certain point in space. \par
The \emph{correlation coefficient} can be defined as:
\begin{equation*}
\rho(\tau) = \frac{R_{uu}(\mathbf{x},\tau)}{u_{rms}^{2}},
\end{equation*}
where $u_{rms}= \sqrt{\langle u'^{2} \rangle}$, which corresponds to $R_{uu}(\mathbf{x},0)$.\par

A similar definition could be provided to measure the correlation in space:
\begin{equation}
R_{ij}(\mathbf{x},\mathbf{r},t) = \langle u'(\mathbf{x},t) u'(\mathbf{x}+\mathbf{r},t) \rangle.
\label{covariance}
\end{equation}
The equation~\ref{covariance} represent the fluctuating parts of the velocity components correlation, set at distance $\mathbf{r}$, and it is called \emph{covariance}.\par
 



\subsection{Spectral analysis}
In the analysis of a random process, not all the necessary information can be deduced by its PDF. In the previous paragraph, correlation has proved to provide additional details about the relations established between different points in time and space. Introducing the spectral analysis enables the possibily to gain information about the energy distribution among the frequencies. In order to give a frequency description of turbulence to see which are the most energy containing frequencies, the Power Spectral Density has to be considered, and the tool used is the Fourier transform. Fourier introduced the idea to split the signal into different harmonics of different weight, in order to reproduce the signal itself. Hence, the signal domain experiences a shift, from time to frequency domain.
\begin{equation}
F(\omega) = \frac{1}{2\pi} \int_{-\infty}^{+\infty} e^{-i\omega t} f(t) dt.
\label{Fourier:t}
\end{equation}

Since the signals encountered when dealing with turbulence are continuous, a description of how the power is distributed between the different frequencies is useful. The power of a signal $u(t)$ is defined as:
\begin{equation}
P = \lim_{T \to \infty} \frac{1}{T} \int_{0}^{+\infty} u(t) dt.
\label{power:u}
\end{equation}
For numerous signals of interest, the equation~\ref{power:u} can not be casted in Fourier domain using~\ref{Fourier:t}. Therefore, the truncated Fourier transform is introduced:
\begin{equation*}
F_{T}(\omega) = \frac{1}{\sqrt{T}} \int_{0}^{+\infty} u(t) e^{i\omega t} dt.
\end{equation*}
The \emph{Power Spectral Density} can be therefore defined as:
\begin{equation*}
S_{uu} = \lim_{T\to \infty} \langle F_{T}(\omega) \rangle.
\end{equation*}
\par
One of the main interesting aspects of spectra analysis is that for a statistically stationary process, PSD constitutes the Fourier transform of the auto-covariance function $R(\mathbf{x},\tau)$:
\begin{equation*}
S_{uu} = \frac{1}{2\pi} \int_{-\infty}^{+\infty} R(\mathbf{x},\tau) e^{i\omega \tau} d\tau
\end{equation*}

with the anti Fourier transform that is:

\begin{equation*}
R(\mathbf{x},\tau) = \int_{-\infty}^{+\infty} S_{uu} e^{i\omega t} d \omega.
\end{equation*}
\par
Since $u$ and $R(\mathbf{x},\tau)$ are both real-valued functions, their Fourier transform is an even function. Hence, $S_{uu}(\omega) = S_{uu}(-\omega)$. Here, a one-sided PSD in the positive frequency is considered
\begin{equation*}
P_{uu} = 2 S_{uu}(\omega)
\end{equation*}
If $\omega$ is positive, 0 otherwise.
\par
Frequently, it is preferred to compute the pre-multiplied one-dimensional energy spectra:
\begin{equation}
\Phi_{uu}^{+} = \frac{\alpha \Phi_{uu}}{u_{\tau}^{2}}
\label{psd}
\end{equation}
with $\Phi_{uu}$ the power spectral density of the streamwise velocity component, $\alpha$ is the streamwise wavenumber and $u_{\tau}$ the friction velocity. In this way, a more global view is allowed.
The same definition of~\ref{psd} applies also for the other two directions.




\subsection{Kolmogorov scales and viscous length scales}
In the previous sections we introduced the concept of scales. However, to completely understand the energy cascade process characteristic of turbulence, we have to define them properly.
\par
The large integral scales are the ones limited by the geometrical dimensions of the considered object. On the other hand the smallest scales are assumed to be independent of the outer geometrical restrictions and depend only on the viscosity and the viscous dissipation itself. They are referred to as the Kolmogorov scales and denoted as: length scale $\eta$, timescale $\tau_{\eta}$ and velocity scale $v_{\eta}$. 
\par
They are recovered through dimensional analysis, assuming independence among viscous dissipation $\epsilon$ and viscosity $\nu$, we can define the Kolmogorov scales as:
\begin{subequations}
\begin{align}
\eta = \sqrt[4]{ \frac{\nu^{3}}{\epsilon} }\\
\tau_{\eta} =\sqrt{ \frac{\nu}{\epsilon}}\\
u_{\eta} = \sqrt[4]{\nu \epsilon }
\end{align}
\end{subequations}

The energy cascade we talked few rows ago is the process which allows energy to transfer from the large sized scales, with their big whirls, to small scales. During such process the bigger whirls put in movement the flow, which generate smaller whirls and so on until the Kolmogorov microscales are reached. At such dimension the viscosity prevails against fluid motion, slowing it down and dissipating the energy through heat. \\~\par

Since close to the wall the main parameters of interest are the wall shear stress, $\tau_{w}$, and the kinematic viscosity, $\nu=\frac{\mu}{\rho}$, it is common practice to use values normalized on those quantities.\par
On this purpose we introduce the \emph{friction velocity} as:
\begin{equation*}
u_{\tau} = \sqrt{\frac{\tau_{w}}{\rho}},
\end{equation*}
and the \emph{viscous length scale} as:
\begin{equation*}
\delta_{\nu} = \frac{\nu}{u_{\tau}}
\end{equation*}
Combining them it is possible to define the $Re_{\tau}$ as:
\begin{equation*}
Re_{\tau} = \frac{u_{\tau}\delta_{\nu}}{\nu}
\end{equation*}
Starting from the definition of the \emph{viscous length scale} we can define the \emph{wall length scale}. \par
The \emph{wall length scale}, defined as
\begin{equation*}
y^{+} = \frac{y}{\delta_{\nu}},
\end{equation*}
is fundamental to determine the turbulence layers. Those layers divide the near wall regions, based on turbulence generation mechanism.
Starting from the wall, we find:
\begin{itemize}
\item the \emph{viscous sublayer} $(y^{+}<5)$, here the entire generation mechanism rely exclusively on viscous effects;
\item the \emph{buffer layer} $(5<y^{+}<30)$, where the turbulence generation is due to the overlapping effects of viscosity and Reynolds stresses;
\item the \emph{log-law region} $(30<y^{+}<0.1y)$, in this region we face the arise of the homonym law, that will be discussed later in detail;
\item the \emph{outer-layer} $(y^{+}>50)$, in this region the Reynolds stresses moves the turbulence engine.
\end{itemize}
